%%%%%%%%%%%%%%%%%%%%%%%%%%%%%%%%%%%%%%%%%
% Beamer Presentation
% LaTeX Template
% Version 1.0 (10/11/12)
%
% This template has been downloaded from:
% http://www.LaTeXTemplates.com
%
% License:
% CC BY-NC-SA 3.0 (http://creativecommons.org/licenses/by-nc-sa/3.0/)
%
%%%%%%%%%%%%%%%%%%%%%%%%%%%%%%%%%%%%%%%%%

%----------------------------------------------------------------------------------------
%	PACKAGES AND THEMES
%----------------------------------------------------------------------------------------

\documentclass{beamer}

\mode<presentation> {

% The Beamer class comes with a number of default slide themes
% which change the colors and layouts of slides. Below this is a list
% of all the themes, uncomment each in turn to see what they look like.

%\usetheme{default}
%\usetheme{AnnArbor}
%\usetheme{Antibes}
%\usetheme{Bergen}
%\usetheme{Berkeley}
%\usetheme{Berlin}
%\usetheme{Boadilla}
%\usetheme{CambridgeUS}
\usetheme{Copenhagen}
%\usetheme{Darmstadt}
%\usetheme{Dresden}
%\usetheme{Frankfurt}
%\usetheme{Goettingen}
%\usetheme{Hannover}
%\usetheme{Ilmenau}
%\usetheme{JuanLesPins}
%\usetheme{Luebeck}
%\usetheme{Madrid}
%\usetheme{Malmoe}
%\usetheme{Marburg}
%\usetheme{Montpellier}
%\usetheme{PaloAlto}
%\usetheme{Pittsburgh}
%\usetheme{Rochester}
%\usetheme{Singapore}
%\usetheme{Szeged}
%\usetheme{Warsaw}

% As well as themes, the Beamer class has a number of color themes
% for any slide theme. Uncomment each of these in turn to see how it
% changes the colors of your current slide theme.

%\usecolortheme{albatross}
%\usecolortheme{beaver}
%\usecolortheme{beetle}
%\usecolortheme{crane}
%\usecolortheme{dolphin}
%\usecolortheme{dove}
%\usecolortheme{fly}
%\usecolortheme{lily}
%\usecolortheme{orchid}
%\usecolortheme{rose}
%\usecolortheme{seagull}
%\usecolortheme{seahorse}
%\usecolortheme{whale}
%\usecolortheme{wolverine}

%\setbeamertemplate{footline} % To remove the footer line in all slides uncomment this line
%\setbeamertemplate{footline}[page number] % To replace the footer line in all slides with a simple slide count uncomment this line

%\setbeamertemplate{navigation symbols}{} % To remove the navigation symbols from the bottom of all slides uncomment this line
}

\usepackage{graphicx} % Allows including images
\usepackage{booktabs} % Allows the use of \toprule, \midrule and \bottomrule in tables

%----------------------------------------------------------------------------------------
%	TITLE PAGE
%----------------------------------------------------------------------------------------

\title[Physics Simulator]{Elementary Physics Visualization App} % The short title appears at the bottom of every slide, the full title is only on the title page

\author{Asim Ukaye, Vikas Kurapati} % Your name
\institute[IITB] % Your institution as it will appear on the bottom of every slide, may be shorthand to save space
{
Department of Aerospace Engineering \\ % Your institution for the title page
\medskip
Indian Institute of Technology, Bombay % Your email address
}
\date{\today} % Date, can be changed to a custom date

\begin{document}

\begin{frame}
\titlepage % Print the title page as the first slide
\end{frame}

\begin{frame}
\frametitle{Overview} % Table of contents slide, comment this block out to remove it
\tableofcontents % Throughout your presentation, if you choose to use \section{} and \subsection{} commands, these will automatically be printed on this slide as an overview of your presentation
\end{frame}
%------------------------------------------------

\begin{frame}
\section{Abstract}
\frametitle{Abstract}
The aim of the project is to simulate interactively basic elementary physics problems using a GUI to show animations according to the inputs received.
\end{frame}

%------------------------------------------------

\begin{frame}
\section{List of Problems simulated}
\frametitle{List of problems simulated}
After consideration from the previous round of scrutiny, we've decided and simulated the following problems to be shown on the GUI according to the inputs of the user.
\begin{itemize}
\item Projectile Motion
\item Cyclotron motion
\item Particles in a box under gravity.
\item Schrodinger's barrier break problem
\end{itemize}
\end{frame}

\begin{frame}
\frametitle{Brief Description}
\subsection{Projectile Motion}
\begin{itemize}
\item Projectile Motion
\end{itemize}
User inputs are initial velocity, angle of projection with the horizontal, acceleration due to gravity which is set to a default value of 9.81, number of time step divisions which is set to a default of 1000.\\
The output of this will be an animation displayed on the User Interface showing the evolution of path of the particle until it hits the horizontal level.\\
\end{frame}

\begin{frame}
\frametitle{Brief Description}
\subsection{Cyclotron Motion}
\begin{itemize}
\item Cyclotron Motion
\end{itemize}
User inputs are time step length , time of the simulation, charge of the particle set to default of 5, mass set to default of 10, Electric field in the x-y plane set to default of 0 in both directions, Initial magnetic field at the initial point in z-direction set to default of 1, initial position set to default at origin, initial velocity $10^4$ in x-direction and $\nabla B$ in both x and y directions of the z-component of the magnetic field. The constraints are imposed as we are doing the simulation only in two dimensions.\\
The output will be an animation displayed on the user interface showing the path of the particle for the specified time given.
\end{frame}

\begin{frame}
\frametitle{Brief Description}
\subsection{Particles in a Box}
\begin{itemize}
\item Particles in a Box
\end{itemize}
User inputs are number of particles, each particle's initial position inside the box, each particle's initial velocity. \\
The output will be an animation displayed on the user interface showing the path of the particles which move under gravity and collide with each other elastically.
\end{frame}
%------------------------------------------------
\begin{frame}
\frametitle{Brief Description}
\subsection{Schrodinger's Barrier}
\begin{itemize}
\item Schrodinger's Barrier
\end{itemize}
Here, there is not much of a user input but the inputs that can be varied by the user are the barrier height, input wave function. \\
The output is the animation showing the variation in position-space wave function and momentum-space wave function.
\end{frame}
\end{document}
